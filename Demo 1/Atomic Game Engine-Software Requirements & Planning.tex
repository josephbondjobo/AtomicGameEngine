\documentclass[a4paper,12pt]{article}
\addtolength{\oddsidemargin}{-1.cm}
\addtolength{\textwidth}{2cm}
\addtolength{\topmargin}{-3cm}
\addtolength{\textheight}{3.5cm}
\makeindex


\usepackage[pdftex]{graphicx}
\usepackage{makeidx}
\usepackage{float}
\usepackage{hyperref}
\hypersetup{
    colorlinks=true,
    linkcolor=blue,
    filecolor=magenta,      
    urlcolor=cyan,
}


% define the title
\author{Team Chowers}
\title{Demo 1}
\begin{document}
\setlength{\parskip}{6pt}

% generates the title
	\begin{titlepage}
		\begin{center}
			\includegraphics[width=1\textwidth]{./up-logo.jpg}\\[1.5cm] 
			\textsc{\LARGE Department of Computer Science} \\ [.5cm]
			\textsc{\Large COS 301 - Atomic Game Engine} \\ [.5cm]
			\textsc{\Large Demo 1} \\ [.5cm]
			\line(1,0){450}\\[.5cm]
			\huge{\bfseries Software Requirements Specification and Project Planning}\\
			\line(1,0){450}\\[.5cm]
			\textsc{\LARGE Team Chowers}\\ [0.5cm]
			
			
			\textsc{\small Jocelyn Bondjobo (13232852)}\\
			\textsc{\small Sara Masilela 	(10126202)}\\
			\textsc{\small Keorapetse Shiko (12231992)}\\
			\textsc{\small Daniel Malangu (13315120)}\\
			
		\end{center}
	\end{titlepage}
	
\tableofcontents
\thispagestyle{empty}
\footnotesize
\normalsize

\newpage
\section{Introduction}
Atomic Game Engine is a relatively new platform for game development. It is open source and aims to compete with other engines like Unity and Unreal. We use Atomic for all of our game development at the moment and are looking to extend its capabilities.
The client, Johan van Staden from Luma Animation has requested the team to research the implementation to make the Atomic Game Engine work with the VR Technology as currently the Atomic Game Engine does not support any VR Technology. And, make sure ot runs at as high a framerate as possible. The team will also have to build an interactive experince using the new technology as a demo.

\newpage
\section{Vision}
Virtual reality is a going to be a big component of the hard-core game experience in the future and opening this technology up for open-source development is something that we would like to push for.
\newpage
\section{Background}
\begin{itemize}
The project was commissioned based on the fact that currently the Atomic Game Engine does not support any VR Technology. Virtual Reality (VR) Technology is a computer technology that replicates an environment, real or imagined, and simulates a user's physical presence and environment to allow for user interaction which is ideal for gaming experience.  
\end{itemize}

\newpage
\section{Software Architecture}
\subsection{Architecture Requirements}
\subsubsection{Architectural Scope}

\subsubsection{Access Channel Requirements}

\subsubsection{Quality Requirements}

 \begin{itemize}
 	\item[$\bullet$]Performance: 
	The aim of this implementation is to make the game play at as a high framerate as possible. Taking into account of the platform in which the game could be running as well.
 	\item[$\bullet$]Reliability:  

 	\item[$\bullet$]Scalability: 

 	\item[$\bullet$]Security:

 	\item[$\bullet$]Flexibility: 
	Flexibility refers to the ability of a system to be changed dynamically for example by extanding the system with some kind of plugin. Therefore, it is very important ofr a system, as a system that is non flexible is restricted to using technologies that were hard coded into it. The game should be deployable in any Android or IOS platform and must work with great performance while using any VR Technology in it.
	 \item[$\bullet$]Maintainability: 

 	\item[$\bullet$]Auditability:

 	\item[$\bullet$]Integrability:
 	The game engine shoud allow for further integration with other platforms such as the Augmented Reality (AR) Technology.
 	\item[$\bullet$]Usability:

 	 \end{itemize}

\subsubsection{Integration Requirements}
\begin{flushleft}
\textbf{Integration Channels}

\textbf{Quality Requirements Integration}

\end{flushleft}

\subsubsection{Architectural Constraints}

\subsection{Architectural patterns or styles}
\subsection{Architectural tactics or strategies}

\newpage
\subsection{Use of reference architectures}
\subsubsection{Examples of reference architectures}
We will be using the following as reference architectures examples to implement our new integrated feature with VR:
\begin{itemize}
\item Efficient 2D/3D Game Enngine
\item Full C++ source code on GitHub
\item Fast platform deployment provided by the Atomic Editor
\item Standards compliant Javascripts and Typescrippts
\item Google Cardboard
\end{itemize}
\subsection{Access and Integration Channels}
\subsubsection{Access Channels}
\subsubsection{Integration Channels}

\newpage
\subsection{Technologies}	
	\subsubsection{Programming languages}
	
	\subsubsection{Operating Systems}
\newpage
\section{Project Planning and implementation}
	
\subsection{Required functionality}

\subsection{Process specifications}

\newpage
\section{Open Issues}

\subsection{References}

\end{document}
